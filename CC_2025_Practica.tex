\documentclass[11pt]{article}

    \usepackage[breakable]{tcolorbox}
    \usepackage{parskip} % Stop auto-indenting (to mimic markdown behaviour)
    

    % Basic figure setup, for now with no caption control since it's done
    % automatically by Pandoc (which extracts ![](path) syntax from Markdown).
    \usepackage{graphicx}
    % Keep aspect ratio if custom image width or height is specified
    \setkeys{Gin}{keepaspectratio}
    % Maintain compatibility with old templates. Remove in nbconvert 6.0
    \let\Oldincludegraphics\includegraphics
    % Ensure that by default, figures have no caption (until we provide a
    % proper Figure object with a Caption API and a way to capture that
    % in the conversion process - todo).
    \usepackage{caption}
    \DeclareCaptionFormat{nocaption}{}
    \captionsetup{format=nocaption,aboveskip=0pt,belowskip=0pt}

    \usepackage{float}
    \floatplacement{figure}{H} % forces figures to be placed at the correct location
    \usepackage{xcolor} % Allow colors to be defined
    \usepackage{enumerate} % Needed for markdown enumerations to work
    \usepackage{geometry} % Used to adjust the document margins
    \usepackage{amsmath} % Equations
    \usepackage{amssymb} % Equations
    \usepackage{textcomp} % defines textquotesingle
    % Hack from http://tex.stackexchange.com/a/47451/13684:
    \AtBeginDocument{%
        \def\PYZsq{\textquotesingle}% Upright quotes in Pygmentized code
    }
    \usepackage{upquote} % Upright quotes for verbatim code
    \usepackage{eurosym} % defines \euro

    \usepackage{iftex}
    \ifPDFTeX
        \usepackage[T1]{fontenc}
        \IfFileExists{alphabeta.sty}{
              \usepackage{alphabeta}
          }{
              \usepackage[mathletters]{ucs}
              \usepackage[utf8x]{inputenc}
          }
    \else
        \usepackage{fontspec}
        \usepackage{unicode-math}
    \fi

    \usepackage{fancyvrb} % verbatim replacement that allows latex
    \usepackage{grffile} % extends the file name processing of package graphics
                         % to support a larger range
    \makeatletter % fix for old versions of grffile with XeLaTeX
    \@ifpackagelater{grffile}{2019/11/01}
    {
      % Do nothing on new versions
    }
    {
      \def\Gread@@xetex#1{%
        \IfFileExists{"\Gin@base".bb}%
        {\Gread@eps{\Gin@base.bb}}%
        {\Gread@@xetex@aux#1}%
      }
    }
    \makeatother
    \usepackage[Export]{adjustbox} % Used to constrain images to a maximum size
    \adjustboxset{max size={0.9\linewidth}{0.9\paperheight}}

    % The hyperref package gives us a pdf with properly built
    % internal navigation ('pdf bookmarks' for the table of contents,
    % internal cross-reference links, web links for URLs, etc.)
    \usepackage{hyperref}
    % The default LaTeX title has an obnoxious amount of whitespace. By default,
    % titling removes some of it. It also provides customization options.
    \usepackage{titling}
    \usepackage{longtable} % longtable support required by pandoc >1.10
    \usepackage{booktabs}  % table support for pandoc > 1.12.2
    \usepackage{array}     % table support for pandoc >= 2.11.3
    \usepackage{calc}      % table minipage width calculation for pandoc >= 2.11.1
    \usepackage[inline]{enumitem} % IRkernel/repr support (it uses the enumerate* environment)
    \usepackage[normalem]{ulem} % ulem is needed to support strikethroughs (\sout)
                                % normalem makes italics be italics, not underlines
    \usepackage{soul}      % strikethrough (\st) support for pandoc >= 3.0.0
    \usepackage{mathrsfs}
    

    
    % Colors for the hyperref package
    \definecolor{urlcolor}{rgb}{0,.145,.698}
    \definecolor{linkcolor}{rgb}{.71,0.21,0.01}
    \definecolor{citecolor}{rgb}{.12,.54,.11}

    % ANSI colors
    \definecolor{ansi-black}{HTML}{3E424D}
    \definecolor{ansi-black-intense}{HTML}{282C36}
    \definecolor{ansi-red}{HTML}{E75C58}
    \definecolor{ansi-red-intense}{HTML}{B22B31}
    \definecolor{ansi-green}{HTML}{00A250}
    \definecolor{ansi-green-intense}{HTML}{007427}
    \definecolor{ansi-yellow}{HTML}{DDB62B}
    \definecolor{ansi-yellow-intense}{HTML}{B27D12}
    \definecolor{ansi-blue}{HTML}{208FFB}
    \definecolor{ansi-blue-intense}{HTML}{0065CA}
    \definecolor{ansi-magenta}{HTML}{D160C4}
    \definecolor{ansi-magenta-intense}{HTML}{A03196}
    \definecolor{ansi-cyan}{HTML}{60C6C8}
    \definecolor{ansi-cyan-intense}{HTML}{258F8F}
    \definecolor{ansi-white}{HTML}{C5C1B4}
    \definecolor{ansi-white-intense}{HTML}{A1A6B2}
    \definecolor{ansi-default-inverse-fg}{HTML}{FFFFFF}
    \definecolor{ansi-default-inverse-bg}{HTML}{000000}

    % common color for the border for error outputs.
    \definecolor{outerrorbackground}{HTML}{FFDFDF}

    % commands and environments needed by pandoc snippets
    % extracted from the output of `pandoc -s`
    \providecommand{\tightlist}{%
      \setlength{\itemsep}{0pt}\setlength{\parskip}{0pt}}
    \DefineVerbatimEnvironment{Highlighting}{Verbatim}{commandchars=\\\{\}}
    % Add ',fontsize=\small' for more characters per line
    \newenvironment{Shaded}{}{}
    \newcommand{\KeywordTok}[1]{\textcolor[rgb]{0.00,0.44,0.13}{\textbf{{#1}}}}
    \newcommand{\DataTypeTok}[1]{\textcolor[rgb]{0.56,0.13,0.00}{{#1}}}
    \newcommand{\DecValTok}[1]{\textcolor[rgb]{0.25,0.63,0.44}{{#1}}}
    \newcommand{\BaseNTok}[1]{\textcolor[rgb]{0.25,0.63,0.44}{{#1}}}
    \newcommand{\FloatTok}[1]{\textcolor[rgb]{0.25,0.63,0.44}{{#1}}}
    \newcommand{\CharTok}[1]{\textcolor[rgb]{0.25,0.44,0.63}{{#1}}}
    \newcommand{\StringTok}[1]{\textcolor[rgb]{0.25,0.44,0.63}{{#1}}}
    \newcommand{\CommentTok}[1]{\textcolor[rgb]{0.38,0.63,0.69}{\textit{{#1}}}}
    \newcommand{\OtherTok}[1]{\textcolor[rgb]{0.00,0.44,0.13}{{#1}}}
    \newcommand{\AlertTok}[1]{\textcolor[rgb]{1.00,0.00,0.00}{\textbf{{#1}}}}
    \newcommand{\FunctionTok}[1]{\textcolor[rgb]{0.02,0.16,0.49}{{#1}}}
    \newcommand{\RegionMarkerTok}[1]{{#1}}
    \newcommand{\ErrorTok}[1]{\textcolor[rgb]{1.00,0.00,0.00}{\textbf{{#1}}}}
    \newcommand{\NormalTok}[1]{{#1}}

    % Additional commands for more recent versions of Pandoc
    \newcommand{\ConstantTok}[1]{\textcolor[rgb]{0.53,0.00,0.00}{{#1}}}
    \newcommand{\SpecialCharTok}[1]{\textcolor[rgb]{0.25,0.44,0.63}{{#1}}}
    \newcommand{\VerbatimStringTok}[1]{\textcolor[rgb]{0.25,0.44,0.63}{{#1}}}
    \newcommand{\SpecialStringTok}[1]{\textcolor[rgb]{0.73,0.40,0.53}{{#1}}}
    \newcommand{\ImportTok}[1]{{#1}}
    \newcommand{\DocumentationTok}[1]{\textcolor[rgb]{0.73,0.13,0.13}{\textit{{#1}}}}
    \newcommand{\AnnotationTok}[1]{\textcolor[rgb]{0.38,0.63,0.69}{\textbf{\textit{{#1}}}}}
    \newcommand{\CommentVarTok}[1]{\textcolor[rgb]{0.38,0.63,0.69}{\textbf{\textit{{#1}}}}}
    \newcommand{\VariableTok}[1]{\textcolor[rgb]{0.10,0.09,0.49}{{#1}}}
    \newcommand{\ControlFlowTok}[1]{\textcolor[rgb]{0.00,0.44,0.13}{\textbf{{#1}}}}
    \newcommand{\OperatorTok}[1]{\textcolor[rgb]{0.40,0.40,0.40}{{#1}}}
    \newcommand{\BuiltInTok}[1]{{#1}}
    \newcommand{\ExtensionTok}[1]{{#1}}
    \newcommand{\PreprocessorTok}[1]{\textcolor[rgb]{0.74,0.48,0.00}{{#1}}}
    \newcommand{\AttributeTok}[1]{\textcolor[rgb]{0.49,0.56,0.16}{{#1}}}
    \newcommand{\InformationTok}[1]{\textcolor[rgb]{0.38,0.63,0.69}{\textbf{\textit{{#1}}}}}
    \newcommand{\WarningTok}[1]{\textcolor[rgb]{0.38,0.63,0.69}{\textbf{\textit{{#1}}}}}


    % Define a nice break command that doesn't care if a line doesn't already
    % exist.
    \def\br{\hspace*{\fill} \\* }
    % Math Jax compatibility definitions
    \def\gt{>}
    \def\lt{<}
    \let\Oldtex\TeX
    \let\Oldlatex\LaTeX
    \renewcommand{\TeX}{\textrm{\Oldtex}}
    \renewcommand{\LaTeX}{\textrm{\Oldlatex}}
    % Document parameters
    % Document title
    \title{CC\_2025\_Practica}
    
    
    
    
    
    
    
% Pygments definitions
\makeatletter
\def\PY@reset{\let\PY@it=\relax \let\PY@bf=\relax%
    \let\PY@ul=\relax \let\PY@tc=\relax%
    \let\PY@bc=\relax \let\PY@ff=\relax}
\def\PY@tok#1{\csname PY@tok@#1\endcsname}
\def\PY@toks#1+{\ifx\relax#1\empty\else%
    \PY@tok{#1}\expandafter\PY@toks\fi}
\def\PY@do#1{\PY@bc{\PY@tc{\PY@ul{%
    \PY@it{\PY@bf{\PY@ff{#1}}}}}}}
\def\PY#1#2{\PY@reset\PY@toks#1+\relax+\PY@do{#2}}

\@namedef{PY@tok@w}{\def\PY@tc##1{\textcolor[rgb]{0.73,0.73,0.73}{##1}}}
\@namedef{PY@tok@c}{\let\PY@it=\textit\def\PY@tc##1{\textcolor[rgb]{0.24,0.48,0.48}{##1}}}
\@namedef{PY@tok@cp}{\def\PY@tc##1{\textcolor[rgb]{0.61,0.40,0.00}{##1}}}
\@namedef{PY@tok@k}{\let\PY@bf=\textbf\def\PY@tc##1{\textcolor[rgb]{0.00,0.50,0.00}{##1}}}
\@namedef{PY@tok@kp}{\def\PY@tc##1{\textcolor[rgb]{0.00,0.50,0.00}{##1}}}
\@namedef{PY@tok@kt}{\def\PY@tc##1{\textcolor[rgb]{0.69,0.00,0.25}{##1}}}
\@namedef{PY@tok@o}{\def\PY@tc##1{\textcolor[rgb]{0.40,0.40,0.40}{##1}}}
\@namedef{PY@tok@ow}{\let\PY@bf=\textbf\def\PY@tc##1{\textcolor[rgb]{0.67,0.13,1.00}{##1}}}
\@namedef{PY@tok@nb}{\def\PY@tc##1{\textcolor[rgb]{0.00,0.50,0.00}{##1}}}
\@namedef{PY@tok@nf}{\def\PY@tc##1{\textcolor[rgb]{0.00,0.00,1.00}{##1}}}
\@namedef{PY@tok@nc}{\let\PY@bf=\textbf\def\PY@tc##1{\textcolor[rgb]{0.00,0.00,1.00}{##1}}}
\@namedef{PY@tok@nn}{\let\PY@bf=\textbf\def\PY@tc##1{\textcolor[rgb]{0.00,0.00,1.00}{##1}}}
\@namedef{PY@tok@ne}{\let\PY@bf=\textbf\def\PY@tc##1{\textcolor[rgb]{0.80,0.25,0.22}{##1}}}
\@namedef{PY@tok@nv}{\def\PY@tc##1{\textcolor[rgb]{0.10,0.09,0.49}{##1}}}
\@namedef{PY@tok@no}{\def\PY@tc##1{\textcolor[rgb]{0.53,0.00,0.00}{##1}}}
\@namedef{PY@tok@nl}{\def\PY@tc##1{\textcolor[rgb]{0.46,0.46,0.00}{##1}}}
\@namedef{PY@tok@ni}{\let\PY@bf=\textbf\def\PY@tc##1{\textcolor[rgb]{0.44,0.44,0.44}{##1}}}
\@namedef{PY@tok@na}{\def\PY@tc##1{\textcolor[rgb]{0.41,0.47,0.13}{##1}}}
\@namedef{PY@tok@nt}{\let\PY@bf=\textbf\def\PY@tc##1{\textcolor[rgb]{0.00,0.50,0.00}{##1}}}
\@namedef{PY@tok@nd}{\def\PY@tc##1{\textcolor[rgb]{0.67,0.13,1.00}{##1}}}
\@namedef{PY@tok@s}{\def\PY@tc##1{\textcolor[rgb]{0.73,0.13,0.13}{##1}}}
\@namedef{PY@tok@sd}{\let\PY@it=\textit\def\PY@tc##1{\textcolor[rgb]{0.73,0.13,0.13}{##1}}}
\@namedef{PY@tok@si}{\let\PY@bf=\textbf\def\PY@tc##1{\textcolor[rgb]{0.64,0.35,0.47}{##1}}}
\@namedef{PY@tok@se}{\let\PY@bf=\textbf\def\PY@tc##1{\textcolor[rgb]{0.67,0.36,0.12}{##1}}}
\@namedef{PY@tok@sr}{\def\PY@tc##1{\textcolor[rgb]{0.64,0.35,0.47}{##1}}}
\@namedef{PY@tok@ss}{\def\PY@tc##1{\textcolor[rgb]{0.10,0.09,0.49}{##1}}}
\@namedef{PY@tok@sx}{\def\PY@tc##1{\textcolor[rgb]{0.00,0.50,0.00}{##1}}}
\@namedef{PY@tok@m}{\def\PY@tc##1{\textcolor[rgb]{0.40,0.40,0.40}{##1}}}
\@namedef{PY@tok@gh}{\let\PY@bf=\textbf\def\PY@tc##1{\textcolor[rgb]{0.00,0.00,0.50}{##1}}}
\@namedef{PY@tok@gu}{\let\PY@bf=\textbf\def\PY@tc##1{\textcolor[rgb]{0.50,0.00,0.50}{##1}}}
\@namedef{PY@tok@gd}{\def\PY@tc##1{\textcolor[rgb]{0.63,0.00,0.00}{##1}}}
\@namedef{PY@tok@gi}{\def\PY@tc##1{\textcolor[rgb]{0.00,0.52,0.00}{##1}}}
\@namedef{PY@tok@gr}{\def\PY@tc##1{\textcolor[rgb]{0.89,0.00,0.00}{##1}}}
\@namedef{PY@tok@ge}{\let\PY@it=\textit}
\@namedef{PY@tok@gs}{\let\PY@bf=\textbf}
\@namedef{PY@tok@ges}{\let\PY@bf=\textbf\let\PY@it=\textit}
\@namedef{PY@tok@gp}{\let\PY@bf=\textbf\def\PY@tc##1{\textcolor[rgb]{0.00,0.00,0.50}{##1}}}
\@namedef{PY@tok@go}{\def\PY@tc##1{\textcolor[rgb]{0.44,0.44,0.44}{##1}}}
\@namedef{PY@tok@gt}{\def\PY@tc##1{\textcolor[rgb]{0.00,0.27,0.87}{##1}}}
\@namedef{PY@tok@err}{\def\PY@bc##1{{\setlength{\fboxsep}{\string -\fboxrule}\fcolorbox[rgb]{1.00,0.00,0.00}{1,1,1}{\strut ##1}}}}
\@namedef{PY@tok@kc}{\let\PY@bf=\textbf\def\PY@tc##1{\textcolor[rgb]{0.00,0.50,0.00}{##1}}}
\@namedef{PY@tok@kd}{\let\PY@bf=\textbf\def\PY@tc##1{\textcolor[rgb]{0.00,0.50,0.00}{##1}}}
\@namedef{PY@tok@kn}{\let\PY@bf=\textbf\def\PY@tc##1{\textcolor[rgb]{0.00,0.50,0.00}{##1}}}
\@namedef{PY@tok@kr}{\let\PY@bf=\textbf\def\PY@tc##1{\textcolor[rgb]{0.00,0.50,0.00}{##1}}}
\@namedef{PY@tok@bp}{\def\PY@tc##1{\textcolor[rgb]{0.00,0.50,0.00}{##1}}}
\@namedef{PY@tok@fm}{\def\PY@tc##1{\textcolor[rgb]{0.00,0.00,1.00}{##1}}}
\@namedef{PY@tok@vc}{\def\PY@tc##1{\textcolor[rgb]{0.10,0.09,0.49}{##1}}}
\@namedef{PY@tok@vg}{\def\PY@tc##1{\textcolor[rgb]{0.10,0.09,0.49}{##1}}}
\@namedef{PY@tok@vi}{\def\PY@tc##1{\textcolor[rgb]{0.10,0.09,0.49}{##1}}}
\@namedef{PY@tok@vm}{\def\PY@tc##1{\textcolor[rgb]{0.10,0.09,0.49}{##1}}}
\@namedef{PY@tok@sa}{\def\PY@tc##1{\textcolor[rgb]{0.73,0.13,0.13}{##1}}}
\@namedef{PY@tok@sb}{\def\PY@tc##1{\textcolor[rgb]{0.73,0.13,0.13}{##1}}}
\@namedef{PY@tok@sc}{\def\PY@tc##1{\textcolor[rgb]{0.73,0.13,0.13}{##1}}}
\@namedef{PY@tok@dl}{\def\PY@tc##1{\textcolor[rgb]{0.73,0.13,0.13}{##1}}}
\@namedef{PY@tok@s2}{\def\PY@tc##1{\textcolor[rgb]{0.73,0.13,0.13}{##1}}}
\@namedef{PY@tok@sh}{\def\PY@tc##1{\textcolor[rgb]{0.73,0.13,0.13}{##1}}}
\@namedef{PY@tok@s1}{\def\PY@tc##1{\textcolor[rgb]{0.73,0.13,0.13}{##1}}}
\@namedef{PY@tok@mb}{\def\PY@tc##1{\textcolor[rgb]{0.40,0.40,0.40}{##1}}}
\@namedef{PY@tok@mf}{\def\PY@tc##1{\textcolor[rgb]{0.40,0.40,0.40}{##1}}}
\@namedef{PY@tok@mh}{\def\PY@tc##1{\textcolor[rgb]{0.40,0.40,0.40}{##1}}}
\@namedef{PY@tok@mi}{\def\PY@tc##1{\textcolor[rgb]{0.40,0.40,0.40}{##1}}}
\@namedef{PY@tok@il}{\def\PY@tc##1{\textcolor[rgb]{0.40,0.40,0.40}{##1}}}
\@namedef{PY@tok@mo}{\def\PY@tc##1{\textcolor[rgb]{0.40,0.40,0.40}{##1}}}
\@namedef{PY@tok@ch}{\let\PY@it=\textit\def\PY@tc##1{\textcolor[rgb]{0.24,0.48,0.48}{##1}}}
\@namedef{PY@tok@cm}{\let\PY@it=\textit\def\PY@tc##1{\textcolor[rgb]{0.24,0.48,0.48}{##1}}}
\@namedef{PY@tok@cpf}{\let\PY@it=\textit\def\PY@tc##1{\textcolor[rgb]{0.24,0.48,0.48}{##1}}}
\@namedef{PY@tok@c1}{\let\PY@it=\textit\def\PY@tc##1{\textcolor[rgb]{0.24,0.48,0.48}{##1}}}
\@namedef{PY@tok@cs}{\let\PY@it=\textit\def\PY@tc##1{\textcolor[rgb]{0.24,0.48,0.48}{##1}}}

\def\PYZbs{\char`\\}
\def\PYZus{\char`\_}
\def\PYZob{\char`\{}
\def\PYZcb{\char`\}}
\def\PYZca{\char`\^}
\def\PYZam{\char`\&}
\def\PYZlt{\char`\<}
\def\PYZgt{\char`\>}
\def\PYZsh{\char`\#}
\def\PYZpc{\char`\%}
\def\PYZdl{\char`\$}
\def\PYZhy{\char`\-}
\def\PYZsq{\char`\'}
\def\PYZdq{\char`\"}
\def\PYZti{\char`\~}
% for compatibility with earlier versions
\def\PYZat{@}
\def\PYZlb{[}
\def\PYZrb{]}
\makeatother


    % For linebreaks inside Verbatim environment from package fancyvrb.
    \makeatletter
        \newbox\Wrappedcontinuationbox
        \newbox\Wrappedvisiblespacebox
        \newcommand*\Wrappedvisiblespace {\textcolor{red}{\textvisiblespace}}
        \newcommand*\Wrappedcontinuationsymbol {\textcolor{red}{\llap{\tiny$\m@th\hookrightarrow$}}}
        \newcommand*\Wrappedcontinuationindent {3ex }
        \newcommand*\Wrappedafterbreak {\kern\Wrappedcontinuationindent\copy\Wrappedcontinuationbox}
        % Take advantage of the already applied Pygments mark-up to insert
        % potential linebreaks for TeX processing.
        %        {, <, #, %, $, ' and ": go to next line.
        %        _, }, ^, &, >, - and ~: stay at end of broken line.
        % Use of \textquotesingle for straight quote.
        \newcommand*\Wrappedbreaksatspecials {%
            \def\PYGZus{\discretionary{\char`\_}{\Wrappedafterbreak}{\char`\_}}%
            \def\PYGZob{\discretionary{}{\Wrappedafterbreak\char`\{}{\char`\{}}%
            \def\PYGZcb{\discretionary{\char`\}}{\Wrappedafterbreak}{\char`\}}}%
            \def\PYGZca{\discretionary{\char`\^}{\Wrappedafterbreak}{\char`\^}}%
            \def\PYGZam{\discretionary{\char`\&}{\Wrappedafterbreak}{\char`\&}}%
            \def\PYGZlt{\discretionary{}{\Wrappedafterbreak\char`\<}{\char`\<}}%
            \def\PYGZgt{\discretionary{\char`\>}{\Wrappedafterbreak}{\char`\>}}%
            \def\PYGZsh{\discretionary{}{\Wrappedafterbreak\char`\#}{\char`\#}}%
            \def\PYGZpc{\discretionary{}{\Wrappedafterbreak\char`\%}{\char`\%}}%
            \def\PYGZdl{\discretionary{}{\Wrappedafterbreak\char`\$}{\char`\$}}%
            \def\PYGZhy{\discretionary{\char`\-}{\Wrappedafterbreak}{\char`\-}}%
            \def\PYGZsq{\discretionary{}{\Wrappedafterbreak\textquotesingle}{\textquotesingle}}%
            \def\PYGZdq{\discretionary{}{\Wrappedafterbreak\char`\"}{\char`\"}}%
            \def\PYGZti{\discretionary{\char`\~}{\Wrappedafterbreak}{\char`\~}}%
        }
        % Some characters . , ; ? ! / are not pygmentized.
        % This macro makes them "active" and they will insert potential linebreaks
        \newcommand*\Wrappedbreaksatpunct {%
            \lccode`\~`\.\lowercase{\def~}{\discretionary{\hbox{\char`\.}}{\Wrappedafterbreak}{\hbox{\char`\.}}}%
            \lccode`\~`\,\lowercase{\def~}{\discretionary{\hbox{\char`\,}}{\Wrappedafterbreak}{\hbox{\char`\,}}}%
            \lccode`\~`\;\lowercase{\def~}{\discretionary{\hbox{\char`\;}}{\Wrappedafterbreak}{\hbox{\char`\;}}}%
            \lccode`\~`\:\lowercase{\def~}{\discretionary{\hbox{\char`\:}}{\Wrappedafterbreak}{\hbox{\char`\:}}}%
            \lccode`\~`\?\lowercase{\def~}{\discretionary{\hbox{\char`\?}}{\Wrappedafterbreak}{\hbox{\char`\?}}}%
            \lccode`\~`\!\lowercase{\def~}{\discretionary{\hbox{\char`\!}}{\Wrappedafterbreak}{\hbox{\char`\!}}}%
            \lccode`\~`\/\lowercase{\def~}{\discretionary{\hbox{\char`\/}}{\Wrappedafterbreak}{\hbox{\char`\/}}}%
            \catcode`\.\active
            \catcode`\,\active
            \catcode`\;\active
            \catcode`\:\active
            \catcode`\?\active
            \catcode`\!\active
            \catcode`\/\active
            \lccode`\~`\~
        }
    \makeatother

    \let\OriginalVerbatim=\Verbatim
    \makeatletter
    \renewcommand{\Verbatim}[1][1]{%
        %\parskip\z@skip
        \sbox\Wrappedcontinuationbox {\Wrappedcontinuationsymbol}%
        \sbox\Wrappedvisiblespacebox {\FV@SetupFont\Wrappedvisiblespace}%
        \def\FancyVerbFormatLine ##1{\hsize\linewidth
            \vtop{\raggedright\hyphenpenalty\z@\exhyphenpenalty\z@
                \doublehyphendemerits\z@\finalhyphendemerits\z@
                \strut ##1\strut}%
        }%
        % If the linebreak is at a space, the latter will be displayed as visible
        % space at end of first line, and a continuation symbol starts next line.
        % Stretch/shrink are however usually zero for typewriter font.
        \def\FV@Space {%
            \nobreak\hskip\z@ plus\fontdimen3\font minus\fontdimen4\font
            \discretionary{\copy\Wrappedvisiblespacebox}{\Wrappedafterbreak}
            {\kern\fontdimen2\font}%
        }%

        % Allow breaks at special characters using \PYG... macros.
        \Wrappedbreaksatspecials
        % Breaks at punctuation characters . , ; ? ! and / need catcode=\active
        \OriginalVerbatim[#1,codes*=\Wrappedbreaksatpunct]%
    }
    \makeatother

    % Exact colors from NB
    \definecolor{incolor}{HTML}{303F9F}
    \definecolor{outcolor}{HTML}{D84315}
    \definecolor{cellborder}{HTML}{CFCFCF}
    \definecolor{cellbackground}{HTML}{F7F7F7}

    % prompt
    \makeatletter
    \newcommand{\boxspacing}{\kern\kvtcb@left@rule\kern\kvtcb@boxsep}
    \makeatother
    \newcommand{\prompt}[4]{
        {\ttfamily\llap{{\color{#2}[#3]:\hspace{3pt}#4}}\vspace{-\baselineskip}}
    }
    

    
    % Prevent overflowing lines due to hard-to-break entities
    \sloppy
    % Setup hyperref package
    \hypersetup{
      breaklinks=true,  % so long urls are correctly broken across lines
      colorlinks=true,
      urlcolor=urlcolor,
      linkcolor=linkcolor,
      citecolor=citecolor,
      }
    % Slightly bigger margins than the latex defaults
    
    \geometry{verbose,tmargin=1in,bmargin=1in,lmargin=1in,rmargin=1in}
    
    

\begin{document}
    
    \maketitle
    
    

    
    \hypertarget{logo_encit.png}{%
\subsection{\texorpdfstring{\protect\includegraphics{CC_2025_Practica_files/398ea254-5a51-4d41-b547-e7c65669aab9.png}}{LOGO\_ENCIT.png}}\label{logo_encit.png}}

\# \textbf{ Cambio Climático } \#\# \textbf{Práctica: FaIR } \#\#\#
Escuela Nacional de Ciencias de la Tierra \#\#\#\# Semestre 2025-I

\begin{center}\rule{0.5\linewidth}{0.5pt}\end{center}

Formato de entrega: 1 notebook por equipo, subido vía MACTI (Moodle).

\begin{center}\rule{0.5\linewidth}{0.5pt}\end{center}

\begin{longtable}[]{@{}
  >{\raggedright\arraybackslash}p{(\columnwidth - 0\tabcolsep) * \real{0.06}}@{}}
\toprule
\endhead
\#\#\# \textbf{ Ejercicio 1 - Los escenarios (25 puntos) } \\
Los escenarios de emisión son fundamentales para entender las
proyecciones climáticas del futuro. Están hechos de narrativas
socioeconómicas que se utilizan para estimar las emisiones futuras.
¿Cómo se ven las diferentes especies químicas en FaIR para los distintos
escenarios? En este ejercicio vamos a intentar entender mejor los
escenarios a través de analizar las emisiones, forzamientos y
temperatura resultante con FaIR: \\
\#\#\#\# \textbf{ 1. Replique las simulaciones hechas en el notebook 1,
es decir, haga una simulación utilizando todos los escenarios. En este
caso utilice solamente 1 modelo a su elección. } \\
\#\#\#\# \textbf{ 2. Compare las emisiones entre escenarios para los
diferentes especies químicas. Algunas de las especies más importantes
son:
\texttt{{[}\textquotesingle{}CO2\ FFI\textquotesingle{},\textquotesingle{}CO2\ AFOLU\textquotesingle{},\textquotesingle{}CH4\textquotesingle{},\textquotesingle{}Sulfur\textquotesingle{},\textquotesingle{}OC\textquotesingle{},\textquotesingle{}NH3\textquotesingle{},\textquotesingle{}NOx\textquotesingle{},\textquotesingle{}N2O\textquotesingle{}{]}}.
Para 5 especies de la lista, y una especie adicional que no se encuentre
en la lista, seleccione e inspeccione los objetos de emisión y
concentración de cada especie. } \\
\#\#\#\# \textbf{ 2a. Grafique, especie por especie, la serie de tiempo
de emisión para los 8 escenarios. } \\
\#\#\#\# \textbf{ 2b. Explique, utilizando su conocimiento e
investigación sobre los escenarios, la evolución temporal de las
emisiones de estas 6 especies. Relacione su interpretación de las
gráficas con las narrativas de los escenarios. No es estrictamente
necesario que explique especie por especie, pero sí es importante no
escatimar en el detalle de los diferentes escenarios. ¿Qué factores de
los escenarios explican tendencias ascendentes o descendentes en cada
escenario/especie? Si en algún momento lo considera necesario, puede
agrupar los escenarios en grupos que compartan cierta característica,
siempre y cuando dé una razón para esta agrupación. } \\
\#\#\#\# \textbf{ 3. Grafique la serie de tiempo de forzamiento total y
temperatura resultante para los 8 escenarios (cada variable en un solo
objeto de figura). Con su conocimiento del inciso anterior, explique sus
resultados. } \\
\bottomrule
\end{longtable}

    \begin{tcolorbox}[breakable, size=fbox, boxrule=1pt, pad at break*=1mm,colback=cellbackground, colframe=cellborder]
\prompt{In}{incolor}{ }{\boxspacing}
\begin{Verbatim}[commandchars=\\\{\}]

\end{Verbatim}
\end{tcolorbox}

    \#\#\# \textbf{ Ejercicio 2 - El metano versus el dióxido de carbono (
40 puntos) }

Este ejercicio tiene como objetivo comparar los impactos climáticos del
metano (CH\(_4\)) y el dióxido de carbono (CO\(_2\)), dos de los
principales gases de efecto invernadero responsables del calentamiento
global. Mientras que el CO\(_2\) es el gas más abundante y persistente,
el CH\(_4\) tiene un potencial de calentamiento a corto plazo mucho
mayor. La comparación entre estos dos gases es esencial para entender
las diferentes estrategias de mitigación y sus impactos en el sistema
climático. El metano, a pesar de su vida más corta en la atmósfera,
puede contribuir de manera desproporcionada al calentamiento en escalas
temporales más reducidas, lo que lo convierte en un objetivo crucial
para las políticas climáticas inmediatas.

\begin{itemize}
\tightlist
\item
  Utilice un modelo diferente al del ejercicio anterior y al de otros
  ejercicios de la práctica.
\item
  Realice 3 simulaciones con dos escenarios (en total son 6
  simulaciones): ssp585 y el ssp460
\end{itemize}

\hypertarget{gruxe1ficas-a-mostrar-como-resultados}{%
\subsubsection{Gráficas a mostrar como
resultados}\label{gruxe1ficas-a-mostrar-como-resultados}}

\begin{itemize}
\tightlist
\item
  3 gráficas, como en los ejercicios previos, con nombres en los ejes,
  chulas y bonitas.

  \begin{itemize}
  \tightlist
  \item
    Forzamiento radiativo.
  \item
    Anomalía de temperatura
  \item
    El desbalance radiativo en el tope de la atmósfera. (TOA)
  \end{itemize}

  La última variable no la hemos visto, pero se puede buscar escribiendo
  el nombre de una clase de Fair (f) y luego buscando en las opciones.
\end{itemize}

\hypertarget{instrucciones}{%
\subsubsection{Instrucciones}\label{instrucciones}}

\hypertarget{realice-la-simulaciuxf3n-control-que-utilizaruxe1-los-paruxe1metros-por-default-del-escenario-ssp585-y-el-ssp460.-haga-una-gruxe1fica-de-forzamiento-radiativo-para-comparar-entre-ambos-escenarios.}{%
\paragraph{\texorpdfstring{\textbf{ 1. Realice la simulación control,
que utilizará los parámetros por default del escenario: ssp585 y el
ssp460. Haga una gráfica de forzamiento radiativo para comparar entre
ambos escenarios.
}}{ 1. Realice la simulación control, que utilizará los parámetros por default del escenario: ssp585 y el ssp460. Haga una gráfica de forzamiento radiativo para comparar entre ambos escenarios. }}\label{realice-la-simulaciuxf3n-control-que-utilizaruxe1-los-paruxe1metros-por-default-del-escenario-ssp585-y-el-ssp460.-haga-una-gruxe1fica-de-forzamiento-radiativo-para-comparar-entre-ambos-escenarios.}}

\hypertarget{realice-una-segunda-simulaciuxf3n-que-estuxe1-caracterizada-por-una-reducciuxf3n-repentina-del-metano-a-partir-del-auxf1o-2024.-entre-los-auxf1os-2024-y-2030-las-emisiones-de-ch_4-seruxe1n-la-mitad-de-las-originales-de-cada-escenario-y-partir-del-auxf1o-2030-seruxe1n-0.}{%
\paragraph{\texorpdfstring{\textbf{ 2. Realice una segunda simulación
que está caracterizada por una reducción repentina del metano a partir
del año 2024. Entre los años 2024 y 2030 las emisiones de \(CH_4\) serán
la mitad de las originales de cada escenario y partir del año 2030 serán
0.
}}{ 2. Realice una segunda simulación que está caracterizada por una reducción repentina del metano a partir del año 2024. Entre los años 2024 y 2030 las emisiones de CH\_4 serán la mitad de las originales de cada escenario y partir del año 2030 serán 0. }}\label{realice-una-segunda-simulaciuxf3n-que-estuxe1-caracterizada-por-una-reducciuxf3n-repentina-del-metano-a-partir-del-auxf1o-2024.-entre-los-auxf1os-2024-y-2030-las-emisiones-de-ch_4-seruxe1n-la-mitad-de-las-originales-de-cada-escenario-y-partir-del-auxf1o-2030-seruxe1n-0.}}

\hypertarget{realice-una-segunda-simulaciuxf3n-que-estuxe1-caracterizada-por-una-reducciuxf3n-repentina-del-diuxf3xido-de-carbono-a-partir-del-auxf1o-2024.-entre-los-auxf1os-2024-y-2050-las-emisiones-de-co_2-seruxe1n-la-mitad-de-las-originales-de-cada-escenario-y-partir-del-auxf1o-2050-seruxe1n-iguales-a-un-14-del-valor-encontrado-en-las-condiciones-promedio-previas-a-1920.-para-esto-le-seruxe1-uxfatil-encontrar-el-valor-promedio-de-las-emisiones-antes-de-1920-y-guardarlo-como-una-variable.}{%
\paragraph{\texorpdfstring{\textbf{ 3. Realice una segunda simulación
que está caracterizada por una reducción repentina del dióxido de
carbono a partir del año 2024. Entre los años 2024 y 2050 las emisiones
de \(CO_2\) serán la mitad de las originales de cada escenario y partir
del año 2050 serán iguales a un 1/4 del valor encontrado en las
condiciones promedio previas a 1920. Para esto le será útil encontrar el
valor promedio de las emisiones antes de 1920 y guardarlo como una
variable.
}}{ 3. Realice una segunda simulación que está caracterizada por una reducción repentina del dióxido de carbono a partir del año 2024. Entre los años 2024 y 2050 las emisiones de CO\_2 serán la mitad de las originales de cada escenario y partir del año 2050 serán iguales a un 1/4 del valor encontrado en las condiciones promedio previas a 1920. Para esto le será útil encontrar el valor promedio de las emisiones antes de 1920 y guardarlo como una variable. }}\label{realice-una-segunda-simulaciuxf3n-que-estuxe1-caracterizada-por-una-reducciuxf3n-repentina-del-diuxf3xido-de-carbono-a-partir-del-auxf1o-2024.-entre-los-auxf1os-2024-y-2050-las-emisiones-de-co_2-seruxe1n-la-mitad-de-las-originales-de-cada-escenario-y-partir-del-auxf1o-2050-seruxe1n-iguales-a-un-14-del-valor-encontrado-en-las-condiciones-promedio-previas-a-1920.-para-esto-le-seruxe1-uxfatil-encontrar-el-valor-promedio-de-las-emisiones-antes-de-1920-y-guardarlo-como-una-variable.}}

\hypertarget{conteste-las-preguntas}{%
\paragraph{\texorpdfstring{\textbf{ 4. Conteste las preguntas:
}}{ 4. Conteste las preguntas: }}\label{conteste-las-preguntas}}

\hypertarget{describa-las-diferencias-entre-las-6-simulaciones.-quuxe9-sucede-con-el-forzamiento-y-la-temperatura-en-cada-una-no-se-olvide-de-sus-simulaciones-control.}{%
\paragraph{\texorpdfstring{\textbf{ - Describa las diferencias entre las
6 simulaciones. ¿Qué sucede con el forzamiento y la temperatura en cada
una? No se olvide de sus simulaciones control.
}}{ - Describa las diferencias entre las 6 simulaciones. ¿Qué sucede con el forzamiento y la temperatura en cada una? No se olvide de sus simulaciones control. }}\label{describa-las-diferencias-entre-las-6-simulaciones.-quuxe9-sucede-con-el-forzamiento-y-la-temperatura-en-cada-una-no-se-olvide-de-sus-simulaciones-control.}}

\hypertarget{entonces-quuxe9-es-muxe1s-importante-disminuir-a-0-las-emisiones-de-metano-o-disminuir-a-la-mitad-las-emisiones-de-co2-y-luego-a-14}{%
\paragraph{\texorpdfstring{\textbf{ - Entonces qué es más importante,
¿disminuir a 0 las emisiones de metano o disminuir a la mitad las
emisiones de CO2 y luego a 1/4?
}}{ - Entonces qué es más importante, ¿disminuir a 0 las emisiones de metano o disminuir a la mitad las emisiones de CO2 y luego a 1/4? }}\label{entonces-quuxe9-es-muxe1s-importante-disminuir-a-0-las-emisiones-de-metano-o-disminuir-a-la-mitad-las-emisiones-de-co2-y-luego-a-14}}

\hypertarget{su-respuesta-depende-del-escenario-considerado-suxed-o-no}{%
\paragraph{\texorpdfstring{\textbf{ - Su respuesta depende del escenario
considerado, ¿sí o
no?}}{ - Su respuesta depende del escenario considerado, ¿sí o no?}}\label{su-respuesta-depende-del-escenario-considerado-suxed-o-no}}

\hypertarget{al-final-de-alguna-de-sus-simulaciones-el-balance-radiativo-en-el-tope-de-la-atmuxf3sfera-llega-a-un-balance-esto-es-lo-que-esperaruxeda-por-quuxe9}{%
\paragraph{\texorpdfstring{\textbf{ - Al final de alguna de sus
simulaciones, ¿el balance radiativo en el tope de la atmósfera llega a
un balance? ¿Esto es lo que esperaría? ¿por qué?
}}{ - Al final de alguna de sus simulaciones, ¿el balance radiativo en el tope de la atmósfera llega a un balance? ¿Esto es lo que esperaría? ¿por qué? }}\label{al-final-de-alguna-de-sus-simulaciones-el-balance-radiativo-en-el-tope-de-la-atmuxf3sfera-llega-a-un-balance-esto-es-lo-que-esperaruxeda-por-quuxe9}}

\begin{center}\rule{0.5\linewidth}{0.5pt}\end{center}

    \begin{tcolorbox}[breakable, size=fbox, boxrule=1pt, pad at break*=1mm,colback=cellbackground, colframe=cellborder]
\prompt{In}{incolor}{ }{\boxspacing}
\begin{Verbatim}[commandchars=\\\{\}]

\end{Verbatim}
\end{tcolorbox}

    \begin{longtable}[]{@{}
  >{\raggedright\arraybackslash}p{(\columnwidth - 0\tabcolsep) * \real{0.06}}@{}}
\toprule
\endhead
\#\#\# \textbf{ Ejercicio 3 (Elegible: 40 puntos) } \\
\#\#\# \textbf{ Opción A: Geoingeniería por aerosoles. } \\
La geoingeniería por aerosoles, también conocida como geoingeniería
solar o gestión de la radiación solar, es un concepto que involucra la
manipulación deliberada de la atmósfera terrestre para contrarrestar el
calentamiento global y sus efectos asociados. Una de las técnicas
propuestas dentro de la geoingeniería por aerosoles implica la
dispersión de partículas en la estratosfera para reflejar una parte de
la radiación solar entrante y así reducir la cantidad de calor que llega
a la superficie terrestre. \\
\#\#\#\# \textbf{ 1. Realice una simulación con 1 modelo y el escenario
ssp460. Esta será su simulación Control. } \\
\#\#\#\# \textbf{ 2. Haga una simulación igual a la Control pero
represente una situación semejante a la que sucedería si la humanidad
decidiera optar por la geoingenería por aerosoles. Es decir, a través de
la modificación de algún forzamiento o de la emisión de una o más
especies, asemeje los efectos que considere que tendría dicha actividad.
Su objetivo es contrarrestar, con un forzamiento negativo, al
forzamiento positivo de los GEIs en el escenario utilizado. Necesita
generar un forzamiento de la misma magnitud pero signo opuesto al de los
GEIs. Piense en las especies en FaIR que podrían tener un forzamiento
negativo. } \\
\#\#\#\# \textbf{ 3. Muestre a través de varias gráficas, cómo su
experimento tiene un forzamiento negativo tan fuerte que se ralentiza el
calentamiento global por el escenario. Para esto, le podría ser útil
considerar como objetivo que la serie de tiempo de temperatura se
mantenga entre +1.0 y 2.0\(^\circ\) por encima de niveles
pre-industriales hasta el año 2100. } \\
\#\#\#\# \textbf{ 6. Grafique la serie de tiempo de temperatura de los 2
experimentos. Explique sus resultados y discuta las implicaciones de la
geoingeniería por aerosoles. Tal vez le sea útil leer:
https://www.theguardian.com/commentisfree/2021/apr/22/climate-crisis-emergency-earth-day
para explicar qué es el término ``termination shock'' y cómo se podría
relacionar con sus resultados. ¿Qué tendría que suceder en su situación
para que sucediese el ``termination shock''? } \\
\#\#\# \textbf{ Opción B La pena climática } \\
La ``pena climática'' es un término que se refiere al resultado de la
reducción de la emisión de aerosoles, los cuales pueden tener efectos
tanto en la calidad del aire como en el clima. Los aerosoles
atmosféricos tienen, en promedio, un efecto de enfriamiento en el clima
al reflejar la radiación solar de vuelta al espacio, contrarrestando
parcialmente el calentamiento causado por los gases de efecto
invernadero. Por otro lado, algunos aerosoles pueden ser contaminantes
del aire y contribuir a la mala calidad del aire y a problemas de salud
pública. \\
Cuando se reduce la emisión de aerosoles como resultado de políticas
ambientales que buscan mejorar la calidad del aire, como la reducción de
la quema de combustibles fósiles, se podría producir un fenómeno
conocido como ``pena climática''. Esta ``pena'', o castigo sería una
traducción más adecuada, se refiere al hecho de que la disminución de
los aerosoles puede eliminar parte de su efecto de enfriamiento en el
clima, lo que lleva a un aumento neto de la temperatura global. \\
\#\#\#\# \textbf{ 1. Replique las simulaciones hechas en el notebook 1
para 2 modelos con los escenarios `ssp370' y `ssp585' (2x2
simulaciones). } \\
\#\#\#\# \textbf{ 2. Haga 4 simulaciones análogas pero, en cada una,
modifique la emisión de las especies
\texttt{{[}\textquotesingle{}Sulfur\textquotesingle{},\textquotesingle{}OC\textquotesingle{},\textquotesingle{}CO\textquotesingle{},\textquotesingle{}BC\textquotesingle{},\textquotesingle{}NOx\textquotesingle{},\textquotesingle{}NH3\textquotesingle{},\textquotesingle{}VOC\textquotesingle{},\textquotesingle{}N2O\textquotesingle{}{]}}
y vuélvalas 0 a partir del año 2024. Esta simulación será representativa
de emisiones de aerosoles netas 0. Explique por qué reducir estas
emisiones sería equivalente a mejorar la calidad del aire. } \\
\#\#\#\# \textbf{ 3. Grafique en 4 paneles la temperatura superficial
resultado. En cada panel, grafique la simulación control correspondiente
con el experimento sin aerosoles. } \\
\#\#\#\# \textbf{ 4. Calcule la pena climática como la diferencia en la
anomalía de temperatura entre su experimento y su experimento sin
aerosoles para cada caso. Considere el promedio de temperatura en los
últimos diez años (2090-2100) para ambas simulaciones antes de obtener
la diferencia. } \\
\#\#\#\# \textbf{ 5. Explique la magnitud, impacto y repercusión de la
pena climática en sus simulaciones como resultado de la disminución
abrupta y la tendencia hacia emisiones 0 de aerosoles. } \\
\#\#\#\# \textbf{ Extra (+8 puntos). Demuestre si es cierto que los
experimentos del punto 2 son ``emisiones de aerosoles netas 0'' o no.
} \\
\bottomrule
\end{longtable}

\begin{center}\rule{0.5\linewidth}{0.5pt}\end{center}

    \begin{tcolorbox}[breakable, size=fbox, boxrule=1pt, pad at break*=1mm,colback=cellbackground, colframe=cellborder]
\prompt{In}{incolor}{ }{\boxspacing}
\begin{Verbatim}[commandchars=\\\{\}]

\end{Verbatim}
\end{tcolorbox}

    \#\#\# \textbf{ Ejercicio Extra - La sensibilidad del clima ( 25 puntos)
}

Corra el modelo para todos los scenarios, es decir:

\texttt{scenarios\ =\ {[}\textquotesingle{}ssp119\textquotesingle{},\ \textquotesingle{}ssp126\textquotesingle{},\ \textquotesingle{}ssp245\textquotesingle{},\ \textquotesingle{}ssp370\textquotesingle{},\ \textquotesingle{}ssp434\textquotesingle{},\ \textquotesingle{}ssp460\textquotesingle{},\ \textquotesingle{}ssp534-over\textquotesingle{},\ \textquotesingle{}ssp585\textquotesingle{}{]}}

utilizando 6 modelos de su elección.

\hypertarget{realice-las-simulaciones-de-manera-estuxe1ndar-sin-modificar-ninguna-especie-en-el-intervalo-de-tiempo-normal.}{%
\paragraph{\texorpdfstring{\textbf{ 1. Realice las simulaciones de
manera estándar, sin modificar ninguna especie, en el intervalo de
tiempo normal.
}}{ 1. Realice las simulaciones de manera estándar, sin modificar ninguna especie, en el intervalo de tiempo normal. }}\label{realice-las-simulaciones-de-manera-estuxe1ndar-sin-modificar-ninguna-especie-en-el-intervalo-de-tiempo-normal.}}

\hypertarget{calcule-la-sensibilidad-del-clima-lambda-para-cada-modelo.-recuerde-que-definimos-lambda-como-lambdadelta-tdelta-f}{%
\paragraph{\texorpdfstring{\textbf{ 2. Calcule la sensibilidad del clima
\(\lambda\) para cada modelo. Recuerde que definimos \(\lambda\) como:
\[              \lambda=\Delta T/\Delta F       \]
}}{ 2. Calcule la sensibilidad del clima \textbackslash lambda para cada modelo. Recuerde que definimos \textbackslash lambda como:               \textbackslash lambda=\textbackslash Delta T/\textbackslash Delta F        }}\label{calcule-la-sensibilidad-del-clima-lambda-para-cada-modelo.-recuerde-que-definimos-lambda-como-lambdadelta-tdelta-f}}

\hypertarget{a.-para-este-fin-calcule-la-anomaluxeda-de-temperatura-y-el-correspondiente-forzamiento-total-a-finales-de-siglo-en-cada-escenario.-basta-obtener-los-valores-al-final-de-la-simulaciuxf3n-promediando-ambas-variables-entre-2080-y-2100.}{%
\paragraph{\texorpdfstring{\textbf{ 2a. Para este fin, calcule la
anomalía de temperatura, y el correspondiente forzamiento total, a
finales de siglo en cada escenario. Basta obtener los valores al final
de la simulación promediando ambas variables entre 2080 y 2100.
}}{ 2a. Para este fin, calcule la anomalía de temperatura, y el correspondiente forzamiento total, a finales de siglo en cada escenario. Basta obtener los valores al final de la simulación promediando ambas variables entre 2080 y 2100. }}\label{a.-para-este-fin-calcule-la-anomaluxeda-de-temperatura-y-el-correspondiente-forzamiento-total-a-finales-de-siglo-en-cada-escenario.-basta-obtener-los-valores-al-final-de-la-simulaciuxf3n-promediando-ambas-variables-entre-2080-y-2100.}}

\hypertarget{utilizando-estos-valores-haga-para-cada-modelo-por-separado-un-gruxe1fico-de-dispersiuxf3n-f-vs-t-y-calcule-la-pendiente-de-esta-dispersiuxf3n-de-puntos.-la-pendiente-por-definiciuxf3n-es-lambda.}{%
\paragraph{\texorpdfstring{\textbf{ 3. Utilizando estos valores, haga,
para cada modelo por separado, un gráfico de dispersión F vs T, y
calcule la pendiente de esta dispersión de puntos. La pendiente, por
definición, es \(\lambda\).
}}{ 3. Utilizando estos valores, haga, para cada modelo por separado, un gráfico de dispersión F vs T, y calcule la pendiente de esta dispersión de puntos. La pendiente, por definición, es \textbackslash lambda. }}\label{utilizando-estos-valores-haga-para-cada-modelo-por-separado-un-gruxe1fico-de-dispersiuxf3n-f-vs-t-y-calcule-la-pendiente-de-esta-dispersiuxf3n-de-puntos.-la-pendiente-por-definiciuxf3n-es-lambda.}}

\hypertarget{reporte-los-resultados-de-la-pendiente-para-cada-modelo.-responda-obtuvo-los-mismos-valores-para-todos-los-modelos-por-quuxe9-son-parecidos-o-diferentes}{%
\paragraph{\texorpdfstring{\textbf{ 4. Reporte los resultados de la
pendiente para cada modelo. Responda, ¿Obtuvo los mismos valores para
todos los modelos? ¿Por qué son parecidos o diferentes?
}}{ 4. Reporte los resultados de la pendiente para cada modelo. Responda, ¿Obtuvo los mismos valores para todos los modelos? ¿Por qué son parecidos o diferentes? }}\label{reporte-los-resultados-de-la-pendiente-para-cada-modelo.-responda-obtuvo-los-mismos-valores-para-todos-los-modelos-por-quuxe9-son-parecidos-o-diferentes}}

    \begin{tcolorbox}[breakable, size=fbox, boxrule=1pt, pad at break*=1mm,colback=cellbackground, colframe=cellborder]
\prompt{In}{incolor}{ }{\boxspacing}
\begin{Verbatim}[commandchars=\\\{\}]

\end{Verbatim}
\end{tcolorbox}


    % Add a bibliography block to the postdoc
    
    
    
\end{document}
